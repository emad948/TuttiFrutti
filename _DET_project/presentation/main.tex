\documentclass[xcolor=dvipsnames]{beamer}
\usepackage{amsmath}
\usepackage{relsize}
\usetheme{EastLansing} 
% Remove navigation bar
\setbeamertemplate{navigation symbols}{}
%%%%%% Design changes
\setbeamercolor{title}{fg=black, bg=green!50!blue!30!white}
\setbeamercolor{frametitle}{fg=black}
%
\setbeamercolor{block title}{fg=black, bg=green!50!blue!50!white}
\setbeamercolor{block body}{fg=black, bg=green!50!blue!30!white}
%
\setbeamercolor{block title alerted}{fg=black, bg=green!45!blue!50!white}
\setbeamercolor{block body alerted}{fg=black, bg=green!40!blue!30!white}
%
\setbeamercolor{block title example}{fg=black, bg=green!35!blue!50!white}
\setbeamercolor{block body example}{fg=black, bg=green!30!blue!30!white}
%
\setbeamercolor{palette primary}{fg=black, bg=green!50!blue!20!white}
\setbeamercolor{palette secondary}{fg=black, bg=green!60!blue!30!white}
\setbeamercolor{palette tertiary}{fg=black, bg=green!60!blue!40!white}
\setbeamercolor{palette quaternary}{fg=black, bg=green!30!blue!20!white}
%
\useinnertheme{circles}
\setbeamercolor{itemize item}{fg=black}
\setbeamercolor{itemize subitem}{fg=black}
\setbeamercolor{itemize subsubitem}{fg=black}
%
%%%%% Design changes %%%%%


% Tikz package
\usepackage{tikz}
\usetikzlibrary{positioning}
% General info
\title{Fachprojekt - Digital Entertainment Technologies}
\subtitle{WiSe 2021/2022 \\ TU Dortmund}
\author{E. Almsouti, C. Kolbe,  L. Labbert, A. Montag}
\institute{}
\date{01.02.2022}


%%%%%%%%%%%%%%%%%%%%%%%%%%%%%%%%%%%%%%%%%%%%%%%%%%%%%%%%%%%%%%%%%%%%%%
\begin{document}
%%%%%%%%%%%%%%%%%% Slide %%%%%%%%%%%%%%%%%%
\begin{frame}
\titlepage
\end{frame}

%%%%%%%%%%%%%%%%%% Slide %%%%%%%%%%%%%%%%%%
\begin{frame}{TuttiFrutti - Spielidee}
\begin{itemize}
\item Multiplayer Spiel
\item Einfach und schnelle Runden mit unterschiedlichen Leveln/Modi \item Vor allem für Freunde-Gruppen gedacht als "party game"
\item Free-for-all 
\end{itemize}

\end{frame}

%%%%%%%%%%%%%%%%%% Slide %%%%%%%%%%%%%%%%%%
\begin{frame}{TuttiFrutti - Infos}
\begin{itemize}
\item  Multiplayer via lokalem Netzwerk oder Steam (keine eigene Game-SteamID) 
\item 3rd Person
\item Ein Spiel besteht aus 4 Runden zwischen 90-150s 
\item Die Punkte einer Runde werden für die Gesamtwertung auf die Spieleranzahl normiert
\item Kein finales Ausscheiden aus einer Lobby möglich
\item Keine hohen Performance-Anforderungen 

\end{itemize}

\end{frame}

%%%%%%%%%%%%%%%%%% Slide %%%%%%%%%%%%%%%%%%
\begin{frame}{TuttiFrutti - Schwerpunkte}
\begin{itemize}
	\item Multiplayer
	\begin{itemize}
		\item IP-basiert über kcp 
		\item Online über Steam
	\end{itemize}
	\item Presentation
	\begin{itemize}
		\item Eigener Player character 
		\item Player-Animations
		\item Particle effects
		\item Animiertes Menu 
		\item Player Konfiguration
	\end{itemize}
\end{itemize}

\end{frame}

%%%%%%%%%%%%%%%%%% Slide %%%%%%%%%%%%%%%%%%
\begin{frame}{Level - HillKing}
\begin{columns}
\begin{column}{0.5\textwidth}
	\begin{itemize}
		\item 3 Plattform abwechselnd 30" lang aktiv sind
		\item Wenn man auf der Plattform steht bekommt man Punkte
		\item Andere Spieler von der Plattform schubsen 
	\end{itemize}
\end{column}
\begin{column}{0.5\textwidth} 
	\begin{center}
		\includegraphics[width=0.9\textwidth]{level_hillking.png}
	\end{center}
\end{column}
\end{columns}

\end{frame}

%%%%%%%%%%%%%%%%%% Slide %%%%%%%%%%%%%%%%%%
\begin{frame}{Level - Crown}
\begin{columns}
\begin{column}{0.5\textwidth}
	\begin{itemize}
		\item Abhängig von der Spieler-Anzahl werden Spieler mit Kronen gespawnt 
		\item Man bekommt Punkte wenn man eine Krone hat -$>$ Kronen stehlen
	\end{itemize}
\end{column}
\begin{column}{0.5\textwidth} 
	\begin{center}
		\includegraphics[width=0.9\textwidth]{level_crown.png}
	\end{center}
\end{column}
\end{columns}

\end{frame}

%%%%%%%%%%%%%%%%%% Slide %%%%%%%%%%%%%%%%%%
\begin{frame}{Level - RunTheLine}
\begin{columns}
\begin{column}{0.5\textwidth}
	\begin{itemize}
		\item Levelbeschreibung + Bilder 
	\end{itemize}
\end{column}
\begin{column}{0.5\textwidth} 
	\begin{center}
		\includegraphics[width=0.9\textwidth]{Level_view.png}
	\end{center}
\end{column}
\end{columns}

\end{frame}

%%%%%%%%%%%%%%%%%% Slide %%%%%%%%%%%%%%%%%%
\begin{frame}{Level - PerfectMatch}
\begin{columns}
\begin{column}{0.5\textwidth}
	\begin{itemize}
		\item Levelbeschreibung + Bilder 
	\end{itemize}
\end{column}
\begin{column}{0.5\textwidth} 
	\begin{center}
		\includegraphics[width=0.9\textwidth]{Level_view.png}
	\end{center}
\end{column}
\end{columns}
\end{frame}

%%%%%%%%%%%%%%%%%% Slide %%%%%%%%%%%%%%%%%%
\begin{frame}{Probleme}
\begin{itemize}
 \item Multiplayer 
 	\begin{itemize}
 		\item Auch mit Framework aufwendig bis alle Details verstanden sind
 		\item Scripte an Synchronisation anpassen macht wenig Spaß
 		\item Testen schwierig/aufwendig, da Fehler oftmals erst nach Host/Client Wechsel, Rundenwechsel, etc. auftauchen
 	\end{itemize}
 \item Probuilder
 	\begin{itemize}
 		\item (Unity-)Koordinaten 
 		\item Snapping local/global
 	\end{itemize}
  \item Git
 	\begin{itemize}
 		\item Zeitaufwendig merge-conflicts auf Grund von Parameter Änderungen auf Objekten durchzugehen
 	\end{itemize}
  \item Blender zusammen mit Unity 
  \item Unity selbst!
\end{itemize}
	
\end{frame}

%%%%%%%%%%%%%%%%%% Slide %%%%%%%%%%%%%%%%%%
\begin{frame}{TuttiFrutti - Planung und Umsetzung}
\begin{itemize}
		\item Zeit-Unterschätzung von Steam (reformulierung )
\end{itemize}
\begin{center}
		\includegraphics[width=0.8\textwidth]{ProjektPlanung_6times90.png}
\end{center}

\end{frame}

%%%%%%%%%%%%%%%%%% Slide %%%%%%%%%%%%%%%%%%
\begin{frame}{Ausblick}
\begin{itemize}
 		\item Tiefere Integration von (Steam)Multiplayer Komponenten
 			\begin{itemize}
 				\item Voice(Chat)
 				\item (Steam-)Achievements
 			\end{itemize}
 		\item Seasons
 			\begin{itemize}
	 			\item Weitere Level + Modi updates
	 			\item Where are the coins?
 			\end{itemize}
 		\item Globales player-ranking
 		\item Weitere Player-Moves, wie Dinge Werfen
\end{itemize}

\end{frame}

%%%%%%%%%%%%%%%%%% Slide %%%%%%%%%%%%%%%%%%
\begin{frame}{Live-Demo}

\begin{center}
	\Huge{\textcolor{OliveGreen}{Let´s play TuttiFrutti!}}
\end{center}

\end{frame}
%%%%%%%%%%%%%%% End of Slides %%%%%%%%%%%%%%%
\end{document}
